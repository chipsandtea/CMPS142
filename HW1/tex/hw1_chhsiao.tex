\documentclass[11pt]{article}
\usepackage{fullpage,amsthm,amsfonts,amssymb,epsfig,amsmath,times,amsthm,enumitem,mathtools,graphicx}

\newtheorem{theorem}{Theorem}
\newtheorem{claim}[theorem]{Claim}
\DeclarePairedDelimiter\ceil{\lceil}{\rceil}
\DeclarePairedDelimiter\floor{\lfloor}{\rfloor}
\graphicspath{{graphs/}}

\begin{document}

\begin{center}
{\bf\Large CMPS 142 --- Spring Quarter 2017 --  Homework 1}\\
{\bf Christopher Hsiao - chhsiao@ucsc.edu - 1398305}
\end{center}

\section*{Solution to Problem 1}

\section*{2 - Linear Regression in Weka}
\textbf{a.)} Root Mean Squared Error: 0.1897\\
\textbf{b.)} For $x = [3, 3, 5]$, our model $-0.1343(x_1) + 1.8477(x_2) + -0.8966(x_3) + 4.3608 = 5.018$\\
\textbf{c.)}
\begin{align}
w = [1, 1, 1, 1] \ \ \eta = 0.1 \\ 
x^{(1)} = [1, 3, 9, 2] \ \ y^{(1)} = 19 \\
x^{(2)} = [1, 6, 9, 1] \ \ y^{(2)} = 19 \\
w_j = w_j + \eta(y^{(i)} - w^Tx^{(i)})x^{(i)}_j
\end{align}
\begin{center}
And using the update rule $(4)$ for each feature $j$ in instance $i$.
\end{center}

Step 1: 
\begin{align*}
w_0 = 1 + 0.1(19 - 15)1 = 1.4\\
w_1 = 1 + 0.1(19 - 15)3 = 2.2\\
w_1 = 1 + 0.1(19 - 15)9 = 4.6\\
w_1 = 1 + 0.1(19 - 15)2 = 1.8\\
w = [1.4, 2.2, 4.6, 1.8]
\end{align*}

Step 2:
\begin{align*}
w_0 = 1.4 + 0.1(19 - 59.6)1 = -2.66\\
w_1 = 2.2 + 0.1(19 - 59.6)6 = -22.16\\
w_2 = 4.6 + 0.1(19 - 59.6)9 = -31.94\\
w_3 = 1.8 + 0.1(19 - 59.6)1= -2.26\\
w = [-2.66, -22.16, -31.94, -2.26]
\end{align*}

\textbf{d.)}
\begin{equation}
\text{Closed Form: } w = (X^TX)^{-1}X^Ty
\end{equation}
\[X =
\begin{bmatrix}
1, 3, 9, 2\\
1, 6, 9, 1\\
1, 7, 7, 7\\
1, 8, 6, 4\\
1, 1, 0, 8
\end{bmatrix}
\ \ y =
\begin{bmatrix}
19\\
19\\
10\\
11\\
-3
\end{bmatrix}
\ \ X^T = 
\begin{bmatrix}
1, 1, 1, 1, 1\\
3, 6, 7, 8, 1\\
9, 9, 7, 6, 0\\
2, 1, 7, 4, 8
\end{bmatrix}
\]
\begin{center}
\pagebreak
Now, we calculate $X^TX \ \text{and} \ X^Ty$
\end{center}
\[X^TX = 
\begin{bmatrix}
5, \ 25, \ 31, \ \ 22\\
25, 159, 178, 101\\
31, 178, 247, 100\\
22, 101, 100, 134
\end{bmatrix}
\ \ x^Ty = 
\begin{bmatrix}
56\\
326\\
478\\
147
\end{bmatrix}
\]
\begin{center}
Putting it all together, we get:
\end{center}
\[w= 
\begin{bmatrix}
4.361\\
-0.134\\
1.848\\
-0.897
\end{bmatrix}\]
\textbf{e.)} //TODO

\end{document}